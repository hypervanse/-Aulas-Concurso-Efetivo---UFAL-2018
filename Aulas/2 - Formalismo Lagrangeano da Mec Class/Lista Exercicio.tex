\documentclass{article}
\usepackage[utf8]{inputenc}
\usepackage{amsmath}
\usepackage{indentfirst}
\usepackage{amssymb}
\usepackage{graphicx}
\usepackage[a4paper, margin=1in]{geometry}

\usepackage{xcolor,sectsty}
\definecolor{astral}{RGB}{46,116,181}
\subsectionfont{\color{astral}}
\sectionfont{\color{astral}}

\linespread{1.15}
\title{\color{astral}\textbf{Lista de Exercícios} \\ \textbf{Formalismo Lagrangeano da Mecânica Clássica}}
\author{Paulo Brandão}
\date{Maio de 2017}

\begin{document}

\maketitle

\noindent 1. Calcule a função $y(x)$ que possui o menor comprimento entre os pontos $(x_1,y_1)$ e $(x_2,y_2)$ utilizando o método variacional.

\vspace{1cm}

\noindent 2. Demonstre as relações (20), (21) e (22) das notas de aula.

\vspace{1cm}

\noindent 3. Generalize os resultados obtidos na seção 5 para o caso de $N$ partículas em 3 dimensões com $M$ coordenadas generalizadas.

\vspace{1cm}

\noindent 4. Encontre as equações do movimento para o sistema descrito no Ex. 3 das notas de aulas.

\vspace{1cm}

\noindent 5. Seja $F = F(q_1,...,q_n)$ uma função arbitrária das coordenadas generalizadas ($q_1,...,q_n$) de um sistema com Lagrangeano $L(q_1,...,q_n,\dot{q}_1,...,\dot{q}_n,t)$. Prove que os dois Lagrangeanos $L$ e $L' = L + dF/dt$ possuem as mesmas equações de movimento.

\end{document}
