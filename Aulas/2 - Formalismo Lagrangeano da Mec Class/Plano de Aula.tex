\documentclass{article}
\usepackage[utf8]{inputenc}
\usepackage{amsmath}
\usepackage{indentfirst}
\usepackage{amssymb}
\usepackage{graphicx}
\usepackage[a4paper, margin=1in]{geometry}

\usepackage{xcolor,sectsty}
\definecolor{astral}{RGB}{46,116,181}
\subsectionfont{\color{astral}}
\sectionfont{\color{astral}}


\linespread{1.15}
\title{\Huge\color{astral}\textbf{Plano de Aula}}
\author{Paulo Brandão}
\date{Maio de 2017}

\begin{document}

\maketitle

\section{Dados de Identificação}

\noindent \textbf{Instituição}: Instituto de Física - Universidade Federal de Alagoas.

\noindent \textbf{Disciplina}: Mecânica Clássica.

\noindent \textbf{Tema}: Formalismo Lagrangeano da Mecânica Clássica.

\noindent \textbf{Professor}: Paulo Cesar Aguiar Brandão Filho.

\noindent \textbf{Tempo total de aula}: 50 minutos.

\noindent \textbf{Data}: 00/00/2018.

\section{Objetivos}

\begin{itemize}
    \item Compreender a base teórica do formalismo Lagrangeano da Mecânica Clássica.
    \item Analisar a base adquirida na resolução de problemas práticos.
\end{itemize}

\section{Conteúdos}

\begin{enumerate}
    \item Introdução
    \item Funções e Funcionais
    \item Princípio de Hamilton
    \item Coordenadas Generalizadas
        \begin{enumerate}
            \item Graus de Liberdade
        \end{enumerate}
    \item Equações de Euler-Lagrange em Coordenadas Generalizadas
    \item Forças de Vínculo
    \item Conclusões
\end{enumerate}

\section{Metodologia e Recursos}

A aula será de caráter expositivo e dialógico, tendo como recursos materiais: quadro, marcador, resumo da aula (anexo 1) e lista de exercícios (anexo 2).

Inicialmente, será destacada a importância de um método variacional aplicado à solução de problemas de origem física e matemática (seção 1). Uma pequena revisão sobre funções e funcionais será feita (seção 2) com o objetivo de nivelar o estudante ao nível matemático do formalismo a ser desenvolvido. Posteriormente, será anunciado o Princípio de Hamilton que forma a base do desenvolvimento variacional da Mecânica de Newton, gerando as equações de Euler-Lagrange (seção 3). A importância das coordenadas generalizadas será introduzida seguida pela definição de graus de liberdade do sistema (seção 4). A generalização das equações de Euler-Lagrange para um sistema descrito por coordenadas arbitrárias será então apresentada (seção 5) e as forças de vínculo (seção 6). O conteúdo programático foi inspirado nas referências [1,2,3,4].

A aula ocorrerá através da promoção contínua da participação do aluno com o que está sendo discutido, estimulando o resgate daquele conhecimento prévio que o mesmo já possa ter adquirido.

\section{Avaliação}

A avaliação será realizada através da participação em aula durante a exposição do tema e do diagnóstico da resolução dos exercícios da lista do anexo II. 

\section{Referências}

\noindent [1] John R. Taylor, \textit{Classical Mechanics} (University Science Books, 2005).

\noindent [2] R. P. Feynman, R. B. Leighton, M. Sands, \textit{The Feynman Lectures on Physics}, Vol. 1 (Addison Wesley, 1977).

\noindent [3] David Morin, \textit{Introduction to Classical Mechanics} (Cambridge University Press, 2008).

\noindent [4] M. Thornthon, J. B. Marion, \textit{Classical Dynamics of Particles and Systems} 5th ed. (Cengage Learning India, 2012).


\end{document}
