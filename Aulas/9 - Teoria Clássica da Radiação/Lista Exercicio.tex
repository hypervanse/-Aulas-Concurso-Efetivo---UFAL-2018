\documentclass{article}
\usepackage[utf8]{inputenc}
\usepackage{amsmath}
\usepackage{indentfirst}
\usepackage{amssymb}
\usepackage{graphicx}
\usepackage[a4paper, margin=1in]{geometry}

\usepackage{xcolor,sectsty}
\definecolor{astral}{RGB}{46,116,181}
\subsectionfont{\color{astral}}
\sectionfont{\color{astral}}

\linespread{1.15}
\title{\color{astral}\textbf{Lista de Exercícios} \\ \textbf{Teoria Clássica da Radiação}}
\author{Paulo Brandão}
\date{Maio de 2017}

\begin{document}

\maketitle

\noindent 1. Mostre que as equações (22) e (23) das notas de aula são soluções das equações da onda (17) e (18) através de substituição direta.

\vspace{1cm}

\noindent 2. Derive a relação (35) das notas de aula para o vetor potencial $\mathbf{A}$.

\vspace{1cm}

\noindent 3. Demonstre os campos elétrico e magnético gerados por uma carga pontual mostrados nas equações (37) e (38) das notas de aula.

\vspace{1cm}

\noindent 4. Encontre os potenciais de Liénard-Wiechert para uma carga pontual que se move na trajetória linear $\mathbf{w}(t) = \mathbf{v}t$ em termos de $(\mathbf{x},t)$.

\vspace{1cm}

\noindent 5. Uma partícula de carga $q$ move-se num círculo de raio $a$ com uma frequência angular constante $\omega$. (Assuma que o círculo pertence ao plano $xy$, centrado na origem, e que em $t=0$ a carga esteja na posição $(x=a,y=0)$. no eixo $x$ positivo). Encontre os potenciais de Liénard-Wiechert para pontos no eixo $z$.

\end{document}
