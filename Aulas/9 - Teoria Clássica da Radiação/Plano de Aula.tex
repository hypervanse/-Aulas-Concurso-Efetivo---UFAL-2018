\documentclass{article}
\usepackage[utf8]{inputenc}
\usepackage{amsmath}
\usepackage{indentfirst}
\usepackage{amssymb}
\usepackage{graphicx}
\usepackage[a4paper, margin=1in]{geometry}

\usepackage{xcolor,sectsty}
\definecolor{astral}{RGB}{46,116,181}
\subsectionfont{\color{astral}}
\sectionfont{\color{astral}}


\linespread{1.0}
\title{\Huge\color{astral}\textbf{Plano de Aula}}
\author{Paulo Brandão}
\date{Maio de 2017}

\begin{document}

\maketitle

\section{Dados de Identificação}

\noindent \textbf{Instituição}: Instituto de Física - Universidade Federal de Alagoas.

\noindent \textbf{Disciplina}: Eletromagnetismo.

\noindent \textbf{Tema}: Teoria Clássica da Radiação.

\noindent \textbf{Professor}: Paulo Cesar Aguiar Brandão Filho.

\noindent \textbf{Tempo total de aula}: 50 minutos.

\noindent \textbf{Data}: 00/00/2018.

\section{Objetivos}

\begin{itemize}
    \item Compreender a base teórica do formalismo da teoria da radiação clássica utilizando as equações de Maxwell.
    \item Analisar a base adquirida na resolução de problemas práticos.
\end{itemize}

\section{Conteúdos}

\begin{enumerate}
    \item Introdução
    \item Transformações de Calibre
        \begin{enumerate}
            \item Potenciais do Campo Eletromagnético
            \item O Calibre de Lorentz
        \end{enumerate}
    \item Potenciais Retardados
    \item Os Potenciais de Liénard-Wiechert
    \item Os Campos de Liénard-Wiechert
    \item O que é Radiação?
    \item Radiação de uma Carga Acelerada com Velocidade Baixa
    \item Radiação de uma Carga Acelerada com a Velocidade e Aceleração Colineares
    \item Discussão Final e Conclusões
\end{enumerate}

\section{Metodologia e Recursos}

A aula será de caráter expositivo e dialógico, tendo como recursos materiais: quadro, marcador, resumo da aula (anexo 1) e lista de exercícios (anexo 2). Inicialmente, será destacada a importância do estudo da radiação do ponto de vista teórico e aplicado, na seção 1. Uma introdução às transformações de Calibre será introduzida na seção 2 com o objetivo de facilitar a análise das soluções não-homogênas das equações de Maxwell. Posteriormente, na seção 3, discutiremos as soluções das equações de Maxwell em termos dos potenciais retardados que formarão a base da discussão de radiação. Neste ponto a aula pode se dividir em duas vertentes, na primeira estudamos o campo gerado por uma carga percorrendo uma dada trajetória e na segunda vertente podemos fazer uma expansão geral em multipolos quando a fonte é periódica. Por conta do nível de graduação da aula, iremos seguir a primeira vertente, iniciando na seção 4, com a introdução dos potenciais de Liénard-Wiechert. Os campos serão estudados na seção 5 e, finalmente a definição de radiação e sua relação com o que foi exposto anteriormente será mostrada na seção 6. As seções 7 e 8 apresentam algumas aplicações do formalismo desenvolvido para alguns casos particulares. Terminaremos nossa aula com a discussão final e as conclusões expostas na seção 9. O conteúdo programático foi inspirado nas referências [1,2,3,4,5,6].

A aula ocorrerá através da promoção contínua da participação do aluno com o que está sendo discutido, estimulando o resgate daquele conhecimento prévio que o mesmo já possa ter adquirido.

\section{Avaliação}

A avaliação será realizada através da participação em aula durante a exposição do tema e do diagnóstico da resolução dos exercícios da lista do anexo II. 

\section{Referências}

\noindent [1] D J. Griffiths, \textit{Introduction to Electrodynamics} (Prentice Hall, 1999).

\noindent [2] R. P. Feynman, R. B. Leighton, M. Sands, \textit{The Feynman Lectures on Physics}, Vol. 2 (Addison Wesley, 1977).

\noindent [3] M. A. Heald, J. B. Marion, \textit{Classical Electromagnetic Radiation}, 3a ed. (Saunders College Publishing, 1995).

\noindent [4] F. Melia, \textit{Electrodynamics} (Chicago Lectures in Physics, 2001).

\noindent [5] W. Greiner, \textit{Classical Electrodynamics} (Springer, 1998).

\noindent [6] J. D. Jackson, \textit{Classical Electrodynamics}, 3a ed. (John Wiley \& Sons, Inc. 1999).


\end{document}
