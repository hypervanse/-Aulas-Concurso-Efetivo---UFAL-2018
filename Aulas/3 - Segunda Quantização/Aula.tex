\documentclass{article}
\usepackage[utf8]{inputenc}
\usepackage{amsmath}
\usepackage{indentfirst}
\usepackage{graphicx,caption}
\usepackage[a4paper, margin=1in]{geometry}
\linespread{1.15}
\usepackage{empheq}
\usepackage[most]{tcolorbox}
\usepackage[margin=3cm]{caption}
\usepackage{siunitx}
\usepackage{array}
\usepackage{braket}
\usepackage{mathtools}

\usepackage{xcolor,sectsty}
\definecolor{astral}{RGB}{46,116,181}
\subsectionfont{\color{astral}}
\sectionfont{\color{astral}}

\title{\includegraphics[width=0.1\textwidth]{ufallogo.png} \\
\Huge{\color{astral}\textbf{Segunda Quantização}}}

%\title{\Huge{\color{astral}\textbf{Sistemas Quânticos Não-Hermitianos}}}

\author{Paulo Brandão}
\date{Maio de 2017}

\newtcbox{\mymath}[1][]{%
    nobeforeafter, math upper, tcbox raise base,
    enhanced, colframe=blue!30!black,
    colback=blue!30, boxrule=1pt,
    #1}
\newcommand*{\bfrac}[2]{\genfrac{\lbrace}{\rbrace}{0pt}{}{#1}{#2}}
\begin{document}

\maketitle

\section{Introdução e Objetivos da Aula}

O estado de uma única partícula cuja dinâmica seja regida pelas leis da mecânica quântica pode ser descrito de várias formas. Uma delas, e é a forma na qual o estudante tem seu primeiro contato, é através da função de onda $\psi$.
















\section{Partículas Distinguíveis e Indistinguíveis}

É sempre possível identificar a trajetória de uma ``partícula clássica'', isto é, uma partícula regida pelas leis da mecânica Newtoniana. Imagine uma piscina com $N$ bolinhas dessas que as crianças se divertem num shopping ou numa festa. Vamos supor que queremos estudar a trajetória da bolinha número 4 que tem uma certa posição inicial $\mathbf{r}(0)$ antes das crianças pularem na piscina. Podemos fazer isso de várias maneiras. Por exemplo, podemos simplesmente pintar o número 4 na bolinha de interesse (de forma que a pintura não interfira na dinâmica) e simplesmente observar sua posição $\mathbf{r}(t)$ no decorrer do tempo. Ao final da festa, ela terá se chocado com várias outras bolinhas de uma forma bastante errática, porém sua trajetória será bem definida e conseguiremos identificar sua posição em qualquer instante de tempo.

Essa identificação da trajetória de uma partícula para qualquer instante de tempo $t$ quando a mesma interage com outros objetos não é possível na descrição quântica do movimento. É bem conhecido que um elétron, por exemplo, é descrito por uma função de onda e que apenas podemos conhecer a probabilidade de encontrá-lo em uma determinada região num certo intervalo de tempo. Pior ainda, a função de onda do elétron pode interagir (se sobrepor) com outras funções de ondas de outros elétrons e a probabilidade de se encontrar o elétron inicial numa certa região perde o sentido. O conceito de \textit{trajetória} não é definido na mecânica quântica quando há partículas interagentes próximas uma das outras (e até mesmo para uma única partícula livre de interações). ``Partículas quânticas'' são completamente indistinguíveis umas das outras (além de possuir a mesma carga, spin, massa, etc). Queremos estudar inicialmente as consequências dessa propriedade fundamental das partículas.

Suponha que $\ket{\Psi}$ seja um ket de estado que representa $n$ partículas idênticas. Então, a função de onda do estado é
\begin{equation}
    \psi(1,2,...,i,...,n) = \braket{1,2,...,i,...,n|\Psi},
\end{equation}
onde utilizamos o índice $i$ para denotar as coordenadas espaciais $\mathbf{r}$ e o spin $s$. Usaremos a fala ``primeira partícula'' para se referir à partícula descrita pelo primeiro argumento de $\psi$, no caso 1, e assim por diante. Assim, $\psi(1,2,...,i,...,n)$ representa a amplitude de se encontrar a partícula 1 na posição $\mathbf{r}_1$ com spin $s_1$, a partícula 2 na posição $\mathbf{r}_2$ com spin $s_2$ e assim por diante. É importante ficar atento ao uso dessas convenções. Por exemplo, considerando duas partículas apenas, $\psi(2,1)$ significa a amplitude de encontrar a partícula 1 na posição $\mathbf{r}_2$ com spin $s_2$ e a partícula 2 na posição $\mathbf{r}_1$ com spin $s_1$. Dizer que as partículas são idênticas significa dizer que não existem interações que possam distingui-las. Assim, qualquer operador correspondente a um observável físico deve tratar todas as partículas de uma maneira equivalente. O Hamiltoniano de $n$ partículas livres de massa $m$, por exemplo, é
\begin{equation}
    H_{0}(1,2,...,n) = \frac{p_1^2}{2m} +\frac{p_2^2}{2m} + ... +\frac{p_n^2}{2m}  
\end{equation}
que é claramente simétrico sob a troca de $\mathbf{p}_i$ com $\mathbf{p}_j$. Em geral, qualquer observável $A(1,2,...,n)$ deve ser uma função simétrica dos índices $1,2,..., n$.


\section{Permutações e Simetria}

É útil introduzir operadores que façam permutações nas partículas. Vamos definir o operador permutação $P_{ij}$ tal que quando ele atua no ket $\ket{\Psi}$, troca o estado $i$ com o $j$. Em termos da função de onda
\begin{equation}
    \braket{1,2,...,i,...j,...,n|P_{ij}|\Psi} = \braket{1,2,...,j,...i,...,n|\Psi}
\end{equation}
ou, utilizando a mesma notaçao $P_{ij}$ para o operador permutação na base posição,
\begin{equation}
    P_{ij}\psi(1,2,...,i,...,j,...,n) = \psi(1,2,...,j,...,i,...,n).
\end{equation}
Em palavras, o operador permutação cria a amplitude de observar a partícula $i$ na posição $\mathbf{r}_j$ com spin $s_j$ e a partícula $j$ na posição $\mathbf{r}_i$ com spin $s_i$. Assim, para um observável $A$ devemos ter
\begin{equation}
    PA = AP \hspace{0.5cm} \text{ou} \hspace{0.5cm} PAP^{-1} = A
\end{equation}
onde $P$ é qualquer permutação. Claramente, a inversa de $P_{ij}$ é $P_{ji}$ como pode ser facilmente verificado.

Suponha que $\ket{\Psi}$ é um autoestado de um Hamiltoniano $H(1,2,...,n)$ com energia $E$. Então, como $H$ comuta com qualquer permutação $P$, temos
\begin{equation}
    HP\ket{\Psi} = PH\ket{\Psi} = PE\ket{\Psi} = EP\ket{\Psi}.
\end{equation}
Assim, $P\ket{\Psi}$ também é um autoestado do Hamiltoniano $H$ com a mesma energia $E$. Se $P\ket{\Phi}$ não é o mesmo estado que $\Psi$ então temos dois estados diferentes com a mesma energia, ou seja, temos degenerescência. Esse fenômeno é conhecido como \textbf{degenerescência de troca} (\textit{exchange degeneracy}). Considere um sistema formado por apenas duas partículas idênticas. Então $P_{12}$ comuta com $H(1,2)$ e podemos procurar os autoestados de $H$ que são autoestados simultâneos de $P_{12}$. Como $P_{12}^{2} = 1$, os autovalores do operador permutação são $\pm 1$. Se $\psi(1,2)$ é qualquer autoestado de $H$ com energia $E$, então
\begin{equation}
    \psi_S(1,2) = \psi(1,2) + \psi(2,1)
\end{equation}
é um autostado de $H$ com a mesma energia $E$ e também é um autoestado de $P_{12}$ com autovalor $+1$:
\begin{equation}
    H\psi_S(1,2) = H\psi(1,2) + H\psi(2,1) = E\psi(1,2) + E\psi(2,1) = E\psi_S(1,2)
\end{equation}
\begin{equation}
    P_{12}\psi_S(1,2) = P_{12}\psi(1,2) + P_{12}\psi(2,1) = \psi(2,1) + \psi(1,2) = (+1)\psi_S(1,2).
\end{equation}
Similarmente, a combinação antissimétrica
\begin{equation}
    \psi_A (1,2) = \psi(1,2) - \psi(2,1)
\end{equation}
é autoestado de $H$ com a mesma energia $E$ e autoestado de $P_{12}$ com autovalor $-1$:
\begin{equation}
    H\psi_A(1,2) = H\psi(1,2) - H\psi(2,1) = E\psi(1,2) - E\psi(2,1) = E\psi_A(1,2)
\end{equation}
\begin{equation}
    P_{12}\psi_A(1,2) = P_{12}\psi(1,2) - P_{12}\psi(2,1) = \psi(2,1) - \psi(1,2) = (-1)\psi_A(1,2).
\end{equation}
Uma das grandes descobertas da física é o fato de que:


\begin{center}

\noindent\rule{16cm}{0.4pt}

\textbf{Um par de partículas idênticas sempre terá uma função de onda que também é um autoestado de} $P_{12}$ \textbf{e, além disso, os autovalores} $\pm 1$ \textbf{dependem apenas dos tipos de partículas envolvidas.}

\noindent\rule{16cm}{0.4pt}

\end{center}
Por exemplo, para um par de elétrons devemos ter uma função de onda do sistema \textbf{sempre} antissimétrica enquanto que para um par de méson $\pi^0$ a função de onda deve ser simétrica. Se ao permutar duas partículas do sistema a função de onda for antissimétrica, dizemos que as partículas são \textbf{férmions}. Partículas cujo estado é simétrico sob a permutação de duas partículas são chamadas de \textbf{bósons}. De fato, existe uma conexão geral entre o spin da partícula e a possível simetria do estado sob permutação:

\begin{center}

\noindent\rule{16cm}{0.4pt}

 \textbf{Partículas com spin inteiro são sempre bósons e partículas com spin fracionário são sempre férmions}.

\noindent\rule{16cm}{0.4pt}

\end{center}
Essa conexão entre spin e a simetria de permutação (que dará origem à estatística de distribuição de partículas nos níveis de energias do sistema) não pode ser provada no contexto da mecânica quântica não-relativística. Na mecânica quântica \textit{relativística} essa conexão é na verdade um teorema que pode ser provado.

Os resultados anteriores se estendem para um sistema formado por $n$ férmions idênticos (isto é, $n$ elétrons ou $n$ prótons)
\begin{equation}
    P_{ij}\psi(1,...,i,...,j,...,n) = -\psi(1,...,i,...,j,...,n)
\end{equation}
e para $n$ bósons idênticos
\begin{equation}
    P_{ij}\psi(1,...,i,...,j,...,n) = +\psi(1,...,i,...,j,...,n).
\end{equation}
Assim, a função de onda para os 47 elétrons num átomo de prata é uma função de onda antissimétrica das coordenadas espaciais e do spin dos elétrons. Observe também que a função de onda não pode mudar sua simetria durante a evolução temporal porque o operador permutação $P_{ij}$ comuta com o Hamiltoniano e é, portanto, uma constante do movimento. Uma função de onda inicialmente antissimétrica permanecerá antissimétrica no decorrer do tempo e o mesmo acontece para uma função de onda inicialmente simétrica.

Temos agora um problema. Quando estudamos as propriedades de um único elétron localizado nas dependências da UFAL, por exemplo, não deveríamos levar em conta o fato de existirem outros elétrons ao redor do mundo e que deveríamos incluir todos eles para descrever a função de onda antissimétrica do sistema? Como é que a função de onda de um único elétron que estudamos no início dos cursos em teoria quântica descreve corretamente a dinâmica do sistema se ignoramos sua função de onda antissimétrica? A resposta é que se não existe praticamente nenhuma sobreposição da função de onda do ``nosso'' elétron com as funções de ondas dos outros elétrons, nós não precisamos nos preocupar em construir uma função de onda antissimétrica. Para demonstrar esse fato de uma maneira mais quantitativa, o estudante deverá resolver os problemas propostos na lista de exercício. A regra é que devemos escrever funções de ondas simétricas ou antissimétricas apenas para as partículas relevantes ao sistema considerado. Se não existe a chance de uma sobreposição entre funções de onda, não precisamos nos preocupar em simetrizar ou não o estado do sistema.

\subsection{Sistemas compostos por férmions e bósons}

Para finalizar a discussão sobre permutação e simetria, devemos discutir um pouco sobre as funções de onda de sistemas com partículas compostas. Considere a função de onda de um sistema formado por dois átomos de Hidrogênio (ignorando os spins)
\begin{equation}
    \psi(\mathbf{r}^e_1,\mathbf{r}^p_1,\mathbf{r}^e_2,\mathbf{r}^p_2),
\end{equation}
onde $\mathbf{r}^e_i$ é a posição do elétron no átomo $i$ e $\mathbf{r}^p_i$ a posição do próton no átomo $i$. A função de onda $\psi$ deve trocar de sinal com a mudança das coordenadas dos dois elétrons
\begin{equation}
    P_{12}^e \psi(\mathbf{r}^e_1,\mathbf{r}^p_1,\mathbf{r}^e_2,\mathbf{r}^p_2) = -\psi(\mathbf{r}^e_1,\mathbf{r}^p_1,\mathbf{r}^e_2,\mathbf{r}^p_2).
\end{equation}
A função de onda também deve mudar de sinal se trocarmos a posição dos dois prótons, pois eles também são férmions
\begin{equation}
    P_{12}^p \psi(\mathbf{r}^e_1,\mathbf{r}^p_1,\mathbf{r}^e_2,\mathbf{r}^p_2) = -\psi(\mathbf{r}^e_1,\mathbf{r}^p_1,\mathbf{r}^e_2,\mathbf{r}^p_2).
\end{equation}
A função de onda, porém, \textbf{não} deve mudar de sinal se trocarmos a posição de um elétron com um próton pois eles não são partículas idênticas. Se trocarmos as posições dos elétrons e dos prótons nos dois átomos de Hidrogênio ficamos com
\begin{equation}
    P_{12}^p P_{12}^e \psi(\mathbf{r}^e_1,\mathbf{r}^p_1,\mathbf{r}^e_2,\mathbf{r}^p_2) = \psi(\mathbf{r}^e_1,\mathbf{r}^p_1,\mathbf{r}^e_2,\mathbf{r}^p_2),
\end{equation}
ou seja, a função de onda do sistema não muda. Assim, átomos de Hidrogênio se comportam como bósons mesmo tendo férmions como partículas constituintes. A regra geral é: Sistemas compostos por um número par de férmions e qualquer número de bósons devem se comportar como bósons, e sistemas compostos por um número ímpar de férmions e qualquer número de bósons devem se comportar como férmions. Por exemplo, átomos de He$^4$, contendo dois prótons, dois nêutrons e dois elétrons são bósons. Átomos de He$^3$, por outro lado, contendo apenas um nêutron são férmions.

\subsection{Princípio da Exclusão de Pauli}

Considere um sistema formado por dois férmions idênticos, 1 e 2. A função de onda (não-normalizada) do sistema é antissimétrica e podemos escrever
\begin{equation}
    \psi_A(1,2) = \psi(1,2) - \psi(2,1)
\end{equation}
fica claro pela relação acima que se as duas partículas ocupam o mesmo estado então $\mathbf{r}_1 = \mathbf{r}_2$ e $s_1 = s_2$ e, consequentemente
\begin{equation}
    \psi_A = 0.
\end{equation}
Para três férmions idênticos
\begin{equation}
    \psi_A(1,2,3) =\frac{1}{\sqrt{3!}}[\psi(1,2,3) - \psi(1,3,2) -\psi(2,1,3) + \psi(2,3,1) - \psi(3,2,1) + \psi(3,1,2)]
\end{equation}
onde $\psi(i,j,k)$ são os autoestados de energia do Hamiltoniano $H(1,2,3)$. Se a partícula 1 e a partícula 2 estão no mesmo estado $a$, então
\begin{equation}
    \psi_A(1,2,3) =\frac{1}{\sqrt{3!}}[\psi(a,a,3) - \psi(a,a,2) -\psi(a,a,3) + \psi(a,a,1) - \psi(a,a,1) + \psi(a,a,2)] = 0.
\end{equation}
A função de onda para o sistema de três férmions idênticos também é nula se as partículas 2 e 3 estiverem no mesmo estado ou se a partículas 1 e 3 estiverem no mesmo estado. O caso geral para $n$ partículas pode ser facilmente deduzido a partir desses dois exemplos com $n = 2$ e $n = 3$. Concluímos então que se dois (ou mais) férmions se encontram no mesmo estado, a função de onda do sistema é nula. Em outras palavras, dois férmions idênticos não podem ocupar o mesmo estado quântico. Esse é o famoso \textbf{princípio da exclusão de Pauli}. Bósons, por outro lado, podem ocupar o mesmo estado quântico com qualquer número de partículas presentes no mesmo estado. O princípio da exclusão de Pauli não se aplica para bósons.


\subsection{Normalização}

Existe uma receita para escrever o estado simétrico ou antissimétrico de $n$ partículas idênticas. 



%\section{Estados de Partículas Idênticas Não-Interagentes}



\section{Segunda Quantização}

Quando lidamos com sistemas que possuem um número muito pequeno de partículas, é relativamente fácil construir uma função de onda simétrica ou assimétrica, dependendo do tipo de simetria global do sistema. Essa tarefa se torna impraticável quando o número de partículas aumenta. Escrever a função de onda de um sistema de 1 mol de átomos de Hidrogênio, por exemplo, é uma tarefa praticamente impossível. Iremos descrever, nas próximas seções, um método muito elegante para tratar a dinâmica desses sistemas que leva em conta a simetria dos estados e dos operadores para sistemas de muitas partículas idênticas. Esse método, que é na verdade uma reformulação da equação de Schrödinger, é chamado de \textit{segunda quantização}.

\subsection{Operadores de Aniquilação e Criação}

Queremos caracterizar o estado de um sistema de partículas idênticas sem utilizar explicitamente a função de onda mas sim pelo número de partículas que ocupam um determinado estado. Em outras palavras, queremos uma nova \textbf{representação} para o estado do sistema físico que incorpore as propriedades de permutação descritas anteriormente e que facilite a análise matemática de alguma forma. É útil separar essa análise em dois casos dependendo se as funções de onda do sistema são simétricas (bósons) ou antissimétricas (férmions).

\subsubsection{Bósons}

Suponha que a resolvemos a equação de Schrödinger para um determinado potencial $V(\mathbf{r})$ e que as soluções estacionárias de uma única partícula são $\{ \phi_0(\mathbf{r})$, $\phi_1(\mathbf{r})$, $\phi_2(\mathbf{r}),...\}$. Se colocarmos $n_0$ bósons idênticos no estado fundamental $\phi_0$ designamos o estado do sistema por $\ket{n_0,0,0...}$. Se adicionarmos mais $n_1$ bósons no estado $\phi_1$, designamos o estado do sistema por $\ket{n_0,n_1,0,...}$ e assim por diante. Para uma distribuição arbitrária de $n_0$ bósons no estado $\phi_0$, $n_1$ bósons no estado $\phi_1$, $n_2$ bósons no estado $\phi_2$, etc, escrevemos
\begin{equation}
    \ket{n_0,n_1,n_2,...}.
\end{equation}
Assim, o estado $\ket{0,0,...}$ representa o vácuo, onde nenhuma partícula se encontra em nenhum estado do sistema e o estado $\ket{1,2,1,0,0,...}$ representa uma partícula no estado $\phi_0$, duas partículas no estado $\phi_1$, uma partícula no estado $\phi_3$ e todos os outros estados estão vazios. Trocamos assim a descrição do sistema através da função de onda pela enumeração do número de partículas que ocupam tais estados. Como bósons não satisfazem o princípio da exclusão de Pauli, podemos ter várias partículas num mesmo estado: $\ket{12,40000,2,0,0...}$.

Vamos definir os operadores de criação e aniquilação da seguinte forma
\begin{equation}
    b_{i}^\dagger \ket{n_0,n_1,...,n_i,...} = \sqrt{n_i + 1}\ket{n_0,n_1,...,n_i+1,...},
\end{equation}
\begin{equation}
    b_{i} \ket{n_0,n_1,...,n_i,...} = \sqrt{n_i}\ket{n_0,n_1,...,n_i-1,...},
\end{equation}
ou seja, $b_i^\dagger$ adiciona um bóson ao estado $\phi_i$ e $b_i$ remove um bóson do estado $\phi_i$. O aluno demonstrará na lista de exercício que uma consequência dessas definições é que
\begin{equation}
    [b_i,b_j^\dagger ] = \delta_{ij} \hspace{0.5cm} [b_i,b_j ] = [b_i^\dagger,b_j^\dagger ] = 0.
\end{equation}
Qualquer estado $\ket{n_0,n_1,...}$ pode ser construído a partir do vácuo pela aplicação dos operadores de criação:
\begin{equation}
    \ket{n_0,n_1,n_2,...} = ... \frac{(b_2^\dagger)^{n_2}}{\sqrt{n_2 !}}\frac{(b_1^\dagger)^{n_1}}{\sqrt{n_1 !}}\frac{(b_0^\dagger)^{n_0}}{\sqrt{n_0 !}}\ket{0,0,0,...}.
\end{equation}

\subsubsection{Férmions}

Para um sistema formado por férmions, a história é um pouco diferente já que eles satisfazem o princípio da exclusão de Pauli. Dois férmions não podem ocupar o mesmo estado. Isso significa que um estado do tipo $\ket{n_0 = 2, n_1 = 1,n_2 = 3,...}$ é \textbf{proibido} pela natureza. Assim, considere apenas dois férmions idênticos. Os únicos estados possíveis em termos de $\ket{n_0,n_1,n_2,..}$ são estados do tipo
\begin{equation}
    \ket{1,1,0,...} \hspace{0.5cm} \ket{1,0,1,...} \hspace{0.5cm} \ket{0,1,1,...} \hspace{0.5cm} \ket{1,0,0,1,...} \hspace{0.5cm} \text{etc...}
\end{equation}
ou seja, $n_i = 0$ ou $1$ para todo $i$. Assim, diferente dos bósons, \textbf{sempre} devemos ter $c^{\dagger 2}_i = c^{2}_i = 0$ para férmions (onde $c$ é o operador de aniquilação e $c^\dagger$ o operador de criação). Com essas propriedades é possível demonstrar que as relações de entre os operadores para férmions são
\begin{equation}
    \{c_i,c_j^\dagger \} = \delta_{ij} \hspace{0.5cm} \{c_i,c_j \} = \{c_i^\dagger,c_j^\dagger \} = 0,
\end{equation}
onde $\{ A,B \} = AB + BA$ é o anticomutador de $A$ e $B$. A grande vantagem de descrever o sistema em termos dos estados $\ket{n_0,n_1,n_2,...}$ e dos operadores de criação e aniquilação é que \textbf{a natureza da relação de comutação satisfaz as propriedades de simetria da função de onda}. Isto é, o estado $\ket{\alpha} = c_1^{\dagger} c_2^\dagger \ket{0,0,...}$, com dois férmions, um no estado $\phi_0$ e o outro no estado $\phi_1$, é \textbf{automaticamente} antissimétrico:
\begin{equation}
    \begin{split}
        P_{12}\ket{\alpha} &= P_{12}c_1^{\dagger} c_2^\dagger \ket{0,0,...} \\
                           &= c_2^{\dagger} c_1^\dagger \ket{0,0,...} \\
                           &= -c_1^{\dagger} c_2^\dagger \ket{0,0,...} \hspace{0.5cm} \text{pois} \hspace{0.5cm} c_1^\dagger c_2^\dagger + c_2^\dagger c_1^\dagger = 0 \\
                           & = -\ket{\alpha}.
    \end{split}
\end{equation}
Da mesma forma, na criação de dois bósons $\ket{\alpha} = b_1^{\dagger} b_2^\dagger \ket{0,0,...}$ é fácil verificar que o estado é simétrico: $P_{12}\ket{\alpha} = +\ket{\alpha}$ (veja lista de exercício).

Em resumo, trocamos a representação através da função de onda simétrica ou antissimétrica $\psi$ de um sistema com $n$ partículas idênticas para a representação através do número de partículas que se encontram em cada estado. A informação da simetrização da função de onda está codificada através das relações de comutação dos operadores de criação e aniquilação para bósons e férmions.


\subsection{Operadores de Campo}

Vamos multiplicar a função de onda $\phi_0$ pelo operador aniquilação do mesmo estado $a_0$, a função de onda $\phi_1$ pelo operador aniquilação $a_1$ e assim por diante
\begin{equation}
    \phi_0(\mathbf{r})a_0 + \phi_1(\mathbf{r})a_1 + \phi_2(\mathbf{r})a_2 + ... = \sum_n \phi_n(\mathbf{r})a_n.
\end{equation}
A expressão acima é uma mistura de operadores que atuam no estado $\ket{n_0,n_1,...}$ e a base $\{ \phi_0(\mathbf{r})$, $\phi_1(\mathbf{r})$, $\phi_2(\mathbf{r}),...\}$. Da mesma forma, podemos formar a relaçao com operadores de criação $a^\dagger$
\begin{equation}
    \phi_0^* (\mathbf{r})a_0^\dagger + \phi_1^* (\mathbf{r})a_1^\dagger + \phi_2^* (\mathbf{r})a_2^\dagger + ... = \sum_n \phi_n^* (\mathbf{r})a_n^\dagger,
\end{equation}
onde foi tomado o complexo conjugado das funções da base. Definimos os \textbf{operadores de campo} $\hat{\psi}(\mathbf{r})$ e $\hat{\psi}^\dagger (\mathbf{r})$ por
\begin{equation}
    \hat{\psi}(\mathbf{r}) = \sum_n \phi_n(\mathbf{r})a_n \hspace{0.5cm} \text{e} \hspace{0.5cm} \hat{\psi}^\dagger (\mathbf{r}) = \sum_n \phi_n^* (\mathbf{r})a_n^\dagger.
\end{equation}
A notação $a$ ou $a^\dagger$ se torna $b$ e $b^\dagger$ ou $c$ e $c^\dagger$ dependendo do sistema considerado. Assim, é possível demonstrar que para bósons temos
\begin{equation}
    [\hat{\psi}(\mathbf{r}),\hat{\psi}^\dagger (\mathbf{r}')] = \delta(\mathbf{r}-\mathbf{r}') \hspace{0.5cm} \text{e} \hspace{0.5cm} [\hat{\psi}(\mathbf{r}),\hat{\psi} (\mathbf{r}')] = [\hat{\psi}^\dagger (\mathbf{r}),\hat{\psi}^\dagger (\mathbf{r}')]=0 \hspace{1cm} \text{(Bósons)}
\end{equation}
e para férmions
\begin{equation}
    \{\hat{\psi}(\mathbf{r}),\hat{\psi}^\dagger (\mathbf{r}')\} = \delta(\mathbf{r}-\mathbf{r}') \hspace{0.5cm} \text{e} \hspace{0.5cm} \{\hat{\psi}(\mathbf{r}),\hat{\psi} (\mathbf{r}')\} = \{\hat{\psi}^\dagger (\mathbf{r}),\hat{\psi}^\dagger (\mathbf{r}')\}=0. \hspace{1cm} \text{(Férmions)}
\end{equation}

Os operadores de criação e aniquilação fornecem uma linguagem extremamente útil para tratar problemas que envolvem um número muito grande de partículas. Todos os operadores físicos podem ser escritos em termos desses operadores.










\end{document}
