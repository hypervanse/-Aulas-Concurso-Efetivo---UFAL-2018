\documentclass{article}
\usepackage[utf8]{inputenc}
\usepackage{amsmath}
\usepackage{indentfirst}
\usepackage{amssymb}
\usepackage{graphicx}
\usepackage[a4paper, margin=1in]{geometry}

\usepackage{xcolor,sectsty}
\definecolor{astral}{RGB}{46,116,181}
\subsectionfont{\color{astral}}
\sectionfont{\color{astral}}
\title{\includegraphics[width=0.1\textwidth]{ufallogo.png} \\
\Huge{\color{astral}\textbf{Plano de Aula}}}

\linespread{1.15}
%\title{\Huge\color{astral}\textbf{Plano de Aula}}
\author{Paulo Brandão}
\date{Maio de 2017}

\begin{document}

\maketitle

\section{Dados de Identificação}

\noindent \textbf{Instituição}: Instituto de Física - Universidade Federal de Alagoas.

\noindent \textbf{Disciplina}: Mecânica Quântica.

\noindent \textbf{Tema}: Segunda Quantização.

\noindent \textbf{Professor}: Paulo Cesar Aguiar Brandão Filho.

\noindent \textbf{Tempo total de aula}: 50 minutos.

\noindent \textbf{Data}: 00/00/2018.

\section{Objetivos}

\begin{itemize}
    \item Compreender a base teórica do formalismo da segunda quantização.
    \item Analisar a base adquirida na resolução de problemas práticos.
\end{itemize}

\section{Conteúdos}

\begin{enumerate}
    \item Introdução e Objetivos
    \item Estados estacionários de uma partícula
    \item $N$ partículas em estados estacionários
    \item Uma nova formulação
        \begin{enumerate}
            \item Operadores de criação e aniquilação para bósons
            \item Operadores de criação e aniquilação para férmions
        \end{enumerate}
    \item Observáveis em termos dos operadores de criação e aniquilação
    \begin{enumerate}
        \item Observáveis de uma partícula
        \item Observáveis de duas partículas
    \end{enumerate}
    \item Operadores de Campo
    \item Conclusões
\end{enumerate}

\section{Metodologia e Recursos}

A aula será de caráter expositivo e dialógico, tendo como recursos materiais: quadro, marcador, resumo da aula (anexo 1) e lista de exercícios (anexo 2).


Discutiremos inicialmente na seção 1 as situações na qual a formulação de Schrödinger da mecânica quântica, através do uso da função de onda, não fornece uma análise adequada do sistema. A seção 2 revisa o conceito de estado estacionário de uma única partícula quântica seguida pela seção 3 onde generalizamos esse conceito para $N$ partículas distribuídas em infinitos estados estacionários. Nesse ponto é discutido o tipo de simetria existente na função de onda do sistema caso as partículas sejam bósons ou férmions idênticos. Na seção 4 discutiremos uma nova formulação para caracterizar o sistema que não faz uso da função de onda de Schrödinger. Operadores de aniquilação e criação de partículas para bósons e férmions serão introduzidos bem como sua relação com as simetrias das funções de onda. A seção 5 tem como objetivo escrever todos os observáveis em termos dos operadores de criação e aniquilação, formando assim a base da segunda quantização. A aula termina na seção 6 com a introdução elementar dos operadores de campo e suas interpretações. A seção 6 apresenta as conclusões. O conteúdo programático foi inspirado nas referências [1,2].

A aula ocorrerá através da promoção contínua da participação do aluno com o que está sendo discutido, estimulando o resgate daquele conhecimento prévio que o mesmo já possa ter adquirido.

\section{Avaliação}

A avaliação será realizada através da participação em aula durante a exposição do tema e do diagnóstico da resolução dos exercícios da lista do anexo II. 

\section{Referências}

\noindent [1] L. D. Landau e E. M. Lifshitz, \textit{Quantum Mechanics - Non-relativistic theory} (Pergamon Press, 1991).

\noindent [2] Gordon Baym, \textit{Lectures on Quantum Mechanics} (Westview Press, 1990).



\end{document}
