\documentclass{article}
\usepackage[utf8]{inputenc}
\usepackage{amsmath}
\usepackage{indentfirst}
\usepackage{amssymb}
\usepackage{graphicx}
\usepackage[a4paper, margin=1in]{geometry}

\usepackage{xcolor,sectsty}
\definecolor{astral}{RGB}{46,116,181}
\subsectionfont{\color{astral}}
\sectionfont{\color{astral}}

\linespread{1.15}
\title{\color{astral} \textbf{Lista de Exercícios} \\ \textbf{Mecânica Não-Linear e Caos}}
\author{Paulo Brandão}
\date{Maio de 2017}

\begin{document}

\maketitle

\noindent 1. Demonstre que todos os operadores que representam portas lógicas, mostrados na aula, são unitários.

\vspace{1cm}

\noindent 2. Calcule a tabela verdade para um circuito quântico formado por duas portas lógicas onde o estado inicial passa primeiro por uma porta controlada Hadamard e, posteriormente, por uma porta CNOT.

\vspace{1cm}

\noindent 3. Construa uma base computacional para um sistema formado por três qubits. Como se comporta a porta QNOT neste caso? E a porta CNOT?

\end{document}
