\documentclass{article}
\usepackage[utf8]{inputenc}
\usepackage{amsmath}
\usepackage{indentfirst}
\usepackage{amssymb}
\usepackage{graphicx}
\usepackage[a4paper, margin=1in]{geometry}
\usepackage{xcolor,sectsty}
\definecolor{astral}{RGB}{46,116,181}
\subsectionfont{\color{astral}}
\sectionfont{\color{astral}}
\linespread{1.15}
\title{\Huge\color{astral}\textbf{Plano de Aula}}
\author{Paulo Brandão}
\date{Maio de 2017}

\begin{document}

\maketitle

\section{Dados de Identificação}

\noindent \textbf{Instituição}: Instituto de Física - Universidade Federal de Alagoas.

\noindent \textbf{Disciplina}: Mecânica Quântica.

\noindent \textbf{Tema}: Informação Quântica.

\noindent \textbf{Professor}: Paulo Cesar Aguiar Brandão Filho.

\noindent \textbf{Tempo total de aula}: 50 minutos.

\noindent \textbf{Data}: 00/00/2018.

\section{Objetivos}

\begin{itemize}
    \item Compreender o processo de armazenamento e transmissão de informação quântica.
    \item Analisar a base adquirida na resolução de problemas práticos.
\end{itemize}

\section{Conteúdos}

\begin{enumerate}
    \item Introdução
    \item Eletrônica digital
    \item Portas lógicas quânticas
        \subitem Porta quântica NOT
        \subitem Porta quântica Hadamard
        \subitem Portas controladas
    \item Circuitos quânticos
    \item Como construir circuitos quânticos?
    \item Conclusões
\end{enumerate}

\section{Metodologia e Recursos}

A aula será de caráter expositivo e dialógico, tendo como recursos materiais: quadro, marcador, resumo da aula (anexo 1) e lista de exercícios (anexo 2).

Inicialmente será discutido na seção 1 as ideias gerais por trás da definição do conceito de informação ou conteúdo informativo de uma mensagem e sua relação com bits e qubits. A seção 2 fornece uma discussão resumida de como bits são traduzidos por um agente físico através de uma sequência de pulsos de voltagem. Discutimos aqui também algumas portas lógicas clássicas que executam tarefas úteis em listas de bits. Na seção 3 começamos o processamento de informação quântica utilizando a base computacional padrão de qubits. As portas lógicas quânticas são introduzidas como operadores unitários que executam tarefas associadas ao estado arbitrário de um sistema de qubits. As propriedades gerais das portas quânticas NOT e Hadamard serão discutidas com mais detalhes. A seção encerra com uma discussão sobre as portas controladas e a importância do seu uso como operações lógicas. A seção 4 discute a ideia de circuitos quânticos formados pelas junções de várias portas e apresenta uma aplicação teórica de um circuito quântico capaz de gerar os estados de Bell (estados emaranhados). Discutimos na seção 5 os principais sistemas físicos candidatos para possíveis implementações das portas quânticas descritas anteriormente. Apresentamos as conclusões na seção 6.  conteúdo programático foi inspirado nas referências [1,2,3,4].

A aula ocorrerá através da promoção contínua da participação do aluno com o que está sendo discutido, estimulando o resgate daquele conhecimento prévio que o mesmo já possa ter adquirido.

\section{Avaliação}

A avaliação será realizada através da participação em aula durante a exposição do tema e do diagnóstico da resolução dos exercícios da lista do anexo II. 

\section{Referências}

\noindent [1] Michael Nielsen and Isaac Chuang, \textit{Quantum computation and quantum information} (AAPT, 2002).

\noindent [2] Stephen M. Barnett, \textit{Quantum Information}, (Oxford University Press, 2009).


\end{document}
