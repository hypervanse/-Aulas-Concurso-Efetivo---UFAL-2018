\documentclass{article}
\usepackage[utf8]{inputenc}
\usepackage{amsmath}
\usepackage{indentfirst}
\usepackage{amssymb}
\usepackage{graphicx}
\usepackage[a4paper, margin=1in]{geometry}

\usepackage{xcolor,sectsty}
\definecolor{astral}{RGB}{46,116,181}
\subsectionfont{\color{astral}}
\sectionfont{\color{astral}}

\linespread{1.15}
\title{\color{astral}\textbf{Lista de Exercícios} \\ \textbf{Teoria Quântica do Espalhamento}}
\author{Paulo Brandão}
\date{Maio de 2017}

\begin{document}

\maketitle


\noindent 1. Encontre a amplitude de espalhamento, na aproximação de Born, para a esfera rígida no limite de baixas energias e calcule sua seção de choque total.

\vspace{1cm}

\noindent 2. Calcule a seção de choque total para o espalhamento por um potencial do tipo Yukawa, na aproximação de Born. Escreva sua resposta em termos da energia $E$.

\vspace{1cm}

\noindent 3. Encontre a função de Green para a equação de Schrödinger em uma dimensão e use-a para construir a forma integral.

\vspace{1cm}

\noindent 4. Calcule a seção de choque diferencial para um potencial do tipo 
\begin{equation}
    V(r) = \beta \frac{e^{-x^2/a^2}}{r}
\end{equation}
onde $\beta$ e $a$ são constantes.
\vspace{1cm}




\end{document}
