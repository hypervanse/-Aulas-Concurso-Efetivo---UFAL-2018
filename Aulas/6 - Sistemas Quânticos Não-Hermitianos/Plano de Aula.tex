\documentclass{article}
\usepackage[utf8]{inputenc}
\usepackage{amsmath}
\usepackage{indentfirst}
\usepackage{amssymb}
\usepackage{graphicx}
\usepackage[a4paper, margin=1in]{geometry}

\usepackage{xcolor,sectsty}
\definecolor{astral}{RGB}{46,116,181}
\subsectionfont{\color{astral}}
\sectionfont{\color{astral}}

\title{\includegraphics[width=0.1\textwidth]{ufallogo.png} \\
\Huge{\color{astral}\textbf{Plano de Aula}}}

\linespread{1.15}
%\title{\Huge\color{astral}\textbf{Plano de Aula}}
\author{Paulo Brandão}
\date{Maio de 2017}

\begin{document}

\maketitle

\section{Dados de Identificação}

\noindent \textbf{Instituição}: Instituto de Física - Universidade Federal de Alagoas.

\noindent \textbf{Disciplina}: Mecânica Quântica.

\noindent \textbf{Tema}: Sistemas Quânticos Não-Hermitianos.

\noindent \textbf{Professor}: Paulo Cesar Aguiar Brandão Filho.

\noindent \textbf{Tempo total de aula}: 50-60 minutos.

\noindent \textbf{Data}: 17/05/2018.

\section{Objetivos}

\begin{itemize}
    \item Compreender o significado físico e matemático de operadores não-Hermitianos com ênfase em Hamiltonianos com simetria $\tilde{P}T$
    \item Analisar a base adquirida na resolução de problemas de teoria quântica.
\end{itemize}

\section{Conteúdos}

\begin{enumerate}
    \item Introdução e Objetivos da Aula
    \item Simetrias de Paridade e Inversão Temporal
    \item O Hamiltoniano $H = P^2 -(iX)^N$
    \subitem 3.1 O Caso $N = 4$ 
    \item Um Aparente Paradoxo
    \item Sistemas Abertos e Fechados
    \item Evolução Temporal de um Sistema $\tilde{P}T$-simétrico.
    \item Aplicações em Sistemas Fotônicos
    \item Conclusões
\end{enumerate}

\section{Metodologia e Recursos}

A aula será de caráter expositivo e dialógico, tendo como recursos materiais: quadro, marcador, resumo da aula (anexo 1) e lista de exercícios (anexo 2).


Será discutido inicialmente na seção 1 a importância da introdução de sistemas descritos por operadores não-hermitianos com ênfase nos postulados usuais da teoria quântica. A seção 2 tem como objetivo nivelar o estudante relembrando os conceitos de simetria estudados numa primeira visão da teoria quântica. Em particular, revisaremos a operação de paridade e de inversão temporal, que desempenharão os papéis principais no restante da aula. A seção 2 tem como objetivo principal ensinar o estudante a identificar operadores que são invariantes sob ação conjunta das operações de paridade e de inversão temporal. Estudaremos na seção 3 o primeiro Hamiltoniano com simetria $\tilde{P}T$ com o objetivo de verificar alguns comportamentos gerais presentes em todos os sistemas com essa simetria. Introduziremos o conceito de quebra espontânea de simetria bem como as fases de um hamiltoniano $\tilde{P}T$-simétrico. Na seção 4 mostraremos uma ``prova'' da realidade dos operadores com simetria $\tilde{P}T$ e discutiremos suas consequências. Sistemas abertos e fechados do ponto de vista dessa simetria serão considerados na seção 5. A evolução temporal dos sistemas será discutida na seção 6 e discutiremos os aspectos mais importantes da primeira implementação experimental de um sistema óptico com simetria $\tilde{P}T$ na seção 6. Por fim, as conclusões serão apresentadas na seçao 8. O conteúdo programático foi inspirado nas referências [1,2,3,4,5].

A aula ocorrerá através da promoção contínua da participação do aluno com o que está sendo discutido, estimulando o resgate daquele conhecimento prévio que o mesmo já possa ter adquirido.

\section{Avaliação}

A avaliação será realizada através da participação em aula durante a exposição do tema e do diagnóstico da resolução dos exercícios da lista do anexo 2. 

\section{Referências}

\noindent [1] Carl M. Bender and S. Boettcher, \textit{Real Spectra in Non-Hermitian Hamiltonians Having $\mathcal{PT}$ symmetry}, Phys. Rev. Lett. \textbf{80} 24 (1998)

\noindent [2] Carl M. Bender, \textit{Upside-down potentials}, Journal of Physics: Conference Series \textbf{343} 012014 (2012).

\noindent [3] A. Guo, G. J. Salamo, D. Duchesne, R. Morandotti, M. Volatier-Ravat, V. Aimez, G. A. Siviloglou and D. N. Christodoulides, \textit{Observation of PT-symmetry breaking in complex optical potentials}, Phys. Rev. Lett. \textbf{103} 023902 (2009).

\noindent [4] J. J. Sakurai, \textit{Modern Quantum Mechanics}, Revised Edition. (Addison-Wesley Publishing Company, Inc, 1994).

\noindent [5] R. Shankar, \textit{Principles of Quantum Mechanics} (Plenum Press, 1994).


\end{document}
