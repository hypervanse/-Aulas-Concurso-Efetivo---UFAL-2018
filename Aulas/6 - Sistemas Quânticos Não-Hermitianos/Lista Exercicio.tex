\documentclass{article}
\usepackage[utf8]{inputenc}
\usepackage{amsmath}
\usepackage{indentfirst}
\usepackage{amssymb}
\usepackage{graphicx}
\usepackage[a4paper, margin=1in]{geometry}
\usepackage{braket}
\usepackage{xcolor,sectsty}
\definecolor{astral}{RGB}{46,116,181}
\subsectionfont{\color{astral}}
\sectionfont{\color{astral}}

\linespread{1.15}
\title{\includegraphics[width=0.1\textwidth]{ufallogo.png} \\
\Huge{\color{astral}\textbf{Anexo 2. \\ Lista de Exercícios}}}
%
%\title{\color{astral}\textbf{Lista de Exercícios} \\ \textbf{Sistemas Quânticos Não-Hermitianos}}


\author{Paulo Brandão}
\date{Maio de 2017}

\begin{document}

\maketitle


\noindent 1. Demonstre que $\tilde{P}\ket{p} = \ket{-p}$.

\vspace{1cm}


\noindent 2. Estude a trajetória \textbf{clássica} de uma partícula descrita pelo Hamiltoniano $H = G^2 - X^4$.

\vspace{1cm}

\noindent 3. Considere três sistemas, $A$, $B$ e $C$ que trocam energia entre eles. O sistema $A$ interage com $B$, o sistema $B$ interage com $C$ e o sistema $C$ interage com $A$. Escreva um hamiltoniano $H_{ABC}$ com simetria PT que descreva a dinâmica desse sistema e calcule seus autovalores.

\vspace{1cm}

\noindent 4. Dê um exemplo de uma distribuição de índice de refração que seja periódica com simetria PT.


\end{document}
