\documentclass{article}
\usepackage[utf8]{inputenc}
\usepackage{amsmath}
\usepackage{indentfirst}
\usepackage{amssymb}
\usepackage{graphicx}
\usepackage[a4paper, margin=1in]{geometry}

\usepackage{xcolor,sectsty}
\definecolor{astral}{RGB}{46,116,181}
\subsectionfont{\color{astral}}
\sectionfont{\color{astral}}

\linespread{1.15}
\title{\color{astral} \textbf{Lista de Exercícios} \\ \textbf{Mecânica Não-Linear e Caos}}
\author{Paulo Brandão}
\date{Maio de 2017}

\begin{document}

\maketitle

\noindent 1. Considere a equação diferencial não-linear de primeira ordem $\dot{x} = 2\sqrt{x-1}$. (a) Por separação de variáveis, encontre a solução $x_1 (t)$. (b) Sua solução deve conter uma constante de integração $k$, e você deve esperar que ela seja a solução geral. Mostre, entretanto, que existe uma outra solução, $x_2 (t) = 1$, que não é da forma $x_1 (t)$ qualquer que seja o valor de $k$. (c) Mostre que embora $x_1 (t)$ e $x_2 (t)$ sejam soluções, nem $Ax_1 (t)$ e $Bx_2 (t)$ ou $x_1 (t) + x_2 (t)$ são soluções. (Isto é, o princípio da superposição não se aplica à equação.) 

\vspace{1cm}

\noindent 2. Aqui está um exemplo diferente das coisas desagradáveis que acontecem com equações não-lineares. Considere a equação não-linear $\dot{x} = 2\sqrt{x}$. Como ela é de primeira ordem, você pode esperar que a especificação de $x(0)$ seja suficiente para determinar uma solução única. Mostre que para essa equação existem dois tipos diferentes de soluções, ambas satisfazendo $x(0) = 0$. Felizmente nenhuma das equações normalmente encontradas em mecânica sofrem dessa desagradável ambiguidade.

\vspace{1cm}

\noindent 3. Resolva numericamente a equação de movimento do PFA e reproduza o gráfico da Figura 3 das notas de aula.

\vspace{1cm}

\noindent 4. Resolva numericamente a equação de movimento do PFA e reproduza o gráfico da Figura 4 das notas de aula.

\vspace{1cm}

\noindent 5. Resolva numericamente a equação de movimento do PFA e reproduza o gráfico da Figura 5 das notas de aula.

\end{document}
