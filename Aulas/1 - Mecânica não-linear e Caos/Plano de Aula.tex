\documentclass{article}
\usepackage[utf8]{inputenc}
\usepackage{amsmath}
\usepackage{indentfirst}
\usepackage{amssymb}
\usepackage{graphicx}
\usepackage[a4paper, margin=1in]{geometry}
\usepackage{xcolor,sectsty}
\definecolor{astral}{RGB}{46,116,181}
\subsectionfont{\color{astral}}
\sectionfont{\color{astral}}
\linespread{1.15}
\title{\Huge\color{astral}\textbf{Plano de Aula}}
\author{Paulo Brandão}
\date{Maio de 2017}

\begin{document}

\maketitle

\section{Dados de Identificação}

\noindent \textbf{Instituição}: Instituto de Física - Universidade Federal de Alagoas.

\noindent \textbf{Disciplina}: Mecânica Clássica.

\noindent \textbf{Tema}: Mecânica Não-Linear e Caos

\noindent \textbf{Professor}: Paulo Cesar Aguiar Brandão Filho.

\noindent \textbf{Tempo total de aula}: 50 minutos.

\noindent \textbf{Data}: 00/00/2018.

\section{Objetivos}

\begin{itemize}
    \item Compreender a base teórica do formalismo da Mecânica Clássica não-linear e sua relação com o caos.
    \item Analisar a base adquirida na resolução de problemas que apresentem comportamento caótico.
\end{itemize}

\section{Conteúdos}

\begin{enumerate}
    \item Introdução
    \item Força Linear e Não-Linear
    \item Modelo Proposto Para Estudar Caos
    \item A Rota Para o Caos
        \begin{enumerate}
            \item Duplicação do Período
        \end{enumerate}
    \item Caos e Condições Iniciais
        \begin{enumerate}
            \item Sensibilidade às Condições Iniciais
            \item O Expoente de Liapunov
            \item O caso $\gamma > \gamma_c$
        \end{enumerate}
    \item Diagramas de Bifurcação
    \item O que vem depois?
\end{enumerate}

\section{Metodologia e Recursos}

A aula será de caráter expositivo e dialógico, tendo como recursos materiais: quadro, marcador, resumo da aula (anexo 1) e lista de exercícios (anexo 2).

Inicialmente, será destacada a importância de forças não-lineares e sua relação com o movimento caótico (seção 1). Uma pequena revisão sobre forças lineares e não-lineares será feita (seção 2) com o objetivo de nivelar o estudante ao nível matemático do formalismo a ser desenvolvido. Posteriormente, será proposto o modelo físico do pêndulo amortecido forçado como um protótipo a ser usado em toda a aula (seção 3). O importante fenômeno da duplicação de período será analisado na seção 4. O estudo do Caos começa na seção 5 onde será demonstrada a sensibilidade do movimento com relação às condições iniciais, seguido por uma apresentação do expoente de Liapunov. Por fim, discutiremos os diagramas de bifurcação que são muito úteis para uma visualização mais adequada do movimento caótico do sistema. A seção 7 discute os próximos temas a serem abordados no curso. O conteúdo programático foi inspirado nas referências [1,2,3,4].

A aula ocorrerá através da promoção contínua da participação do aluno com o que está sendo discutido, estimulando o resgate daquele conhecimento prévio que o mesmo já possa ter adquirido.

\section{Avaliação}

A avaliação será realizada através da participação em aula durante a exposição do tema e do diagnóstico da resolução dos exercícios da lista do anexo II. 

\section{Referências}

\noindent [1] John R. Taylor, \textit{Classical Mechanics} (University Science Books, 2005).

\noindent [2] R. P. Feynman, R. B. Leighton, M. Sands, \textit{The Feynman Lectures on Physics}, Vol. 1 (Addison Wesley, 1977).

\noindent [3] David Morin, \textit{Introduction to Classical Mechanics} (Cambridge University Press, 2008).

\noindent [4] M. Thornthon, J. B. Marion, \textit{Classical Dynamics of Particles and Systems} 5th ed. (Cengage Learning India, 2012).


\end{document}
