\documentclass{article}
\usepackage[utf8]{inputenc}
\usepackage{amsmath}
\usepackage{indentfirst}
\usepackage{amssymb}
\usepackage{graphicx}
\usepackage[a4paper, margin=1in]{geometry}

\usepackage{xcolor,sectsty}
\definecolor{astral}{RGB}{46,116,181}
\subsectionfont{\color{astral}}
\sectionfont{\color{astral}}

\title{\includegraphics[width=0.1\textwidth]{ufallogo.png} \\
\Huge{\color{astral}\textbf{Plano de Aula}}}

\linespread{1.15}
%\title{\Huge\color{astral}\textbf{Plano de Aula}}
\author{Paulo Brandão}
\date{Maio de 2017}

\begin{document}

\maketitle

\section{Dados de Identificação}

\noindent \textbf{Instituição}: Instituto de Física - Universidade Federal de Alagoas.

\noindent \textbf{Disciplina}: Mecânica Quântica.

\noindent \textbf{Tema}: Dissipação em sistemas quânticos.

\noindent \textbf{Professor}: Paulo Cesar Aguiar Brandão Filho.

\noindent \textbf{Tempo total de aula}: 50 minutos.

\noindent \textbf{Data}: 00/00/2018.

\section{Objetivos}

\begin{itemize}
    \item Compreender a origem e o tratamento de processos dissipativos em sistemas quânticos.
    \item Analisar a base adquirida na resolução de problemas de teoria quântica.
\end{itemize}

\section{Conteúdos}

\begin{enumerate}
    \item Introdução
    \item Dissipação em sistemas clássicos
    \item Dissipação em sistemas quânticos
    \subitem 3.1 Por que não quantizar diretamente? 
    \item A equação mestra
    \item Oscilador harmônico quântico amortecido
    \item Emissão espontânea de um átomo de dois níveis
    \item Conclusões
\end{enumerate}

\section{Metodologia e Recursos}

A aula será de caráter expositivo e dialógico, tendo como recursos materiais: quadro, marcador, resumo da aula (anexo 1) e lista de exercícios (anexo 2).

Será discutido inicialmente o papel da dissipação em sistemas clássicos. Essa discussão é importante porque nivela o estudante para os tópicos posteriores e também porque a dissipação em sistemas clássicos nos dá uma ideia de como tratar processos dissipativos em mecânica quântica. Através do sistema oscilador harmônico amortecido, mostraremos posteriormente a inviabilidade de se quantizar esse sistema clássico diretamente transformando as grandezas posição e momento em operadores. A nova abordagem para tratar dissipação em sistemas quânticos será feita através do uso do operador densidade, muito útil para descrever sistemas puros bem como misturas quânticas. A equação mestra será discutida e suas aproximações consideradas. A partir daqui, o aluno deverá perceber a maneira de como deve-se tratar sistemas dissipativos em mecânica quântica, que é o objetivo principal da aula. Para firmar as ideias discutidas anteriormente, duas aplicações da equação mestra serão discutidas. A primeira envolve o processo de dissipação de uma radiação eletromagnética dentro de uma cavidade e a segunda aplicação ilustra a emissão espontânea de um átomo excitado de dois níveis. Por fim, as conclusões serão proferidas. O conteúdo programático foi inspirado na referência [1].

A aula ocorrerá através da promoção contínua da participação do aluno com o que está sendo discutido, estimulando o resgate daquele conhecimento prévio que o mesmo já possa ter adquirido.

\section{Avaliação}

A avaliação será realizada através da participação em aula durante a exposição do tema e do diagnóstico da resolução dos exercícios da lista do anexo II. 

\section{Referências}

\noindent [1] H. J. Carmichael, \textit{Statistical Methods in Quantum Optics 1: Master Equations and Fokker-Planck Equations} (Springer, 1999).


\end{document}
