\documentclass{article}
\usepackage[utf8]{inputenc}
\usepackage{amsmath}
\usepackage{indentfirst}
\usepackage{amssymb}
\usepackage{graphicx}
\usepackage[a4paper, margin=1in]{geometry}
\usepackage{braket}
\usepackage{xcolor,sectsty}
\definecolor{astral}{RGB}{46,116,181}
\subsectionfont{\color{astral}}
\sectionfont{\color{astral}}

\linespread{1.15}
\title{\color{astral}\textbf{Lista de Exercícios} \\ \textbf{Sistemas Quânticos Não-Hermitianos}}


\author{Paulo Brandão}
\date{Maio de 2017}

\begin{document}

\maketitle


\noindent 1. Das equações descritas na aula, deduza:

\begin{itemize}
    \item As relações de comutação:
\end{itemize} 
\begin{equation}
    [\sigma_+ , \sigma_-] = \sigma_z, \hspace{0.5cm} [\sigma_{\pm},\sigma_z] = \mp 2 \sigma_{\pm}
\end{equation}

\begin{itemize}
    \item As ações nos estados atômicos:
\end{itemize} 
\begin{equation}
    \sigma_z \ket{1} = -\ket{1}, \hspace{0.5cm} \sigma_z \ket{2} = \ket{2}
\end{equation}
\begin{equation}
    \sigma_- \ket{1} = 0, \hspace{0.5cm} \sigma_- \ket{2} = \ket{1}
\end{equation}
\begin{equation}
    \sigma_+ \ket{1} = \ket{2}, \hspace{0.5cm} \sigma_+ \ket{2} = 0.
\end{equation}



\vspace{1cm}


\noindent 2. Considere um átomo de dois níveis, como descrito na aula, interagindo com um modo apenas do campo. Calcule a probabilidade $P_2$ de encontrar o átomo no estado excitado e a probabilidade $P_1$ de encontrar o átomo no estado fundamental. Você deve encontrar uma função periódica e não exponencial, como mostrado na aula, pois agora seu sistema é reversível. Essas são as oscilações de Rabi. 

\end{document}
