\documentclass{article}
\usepackage[utf8]{inputenc}
\usepackage{amsmath}
\usepackage{indentfirst}
\usepackage{amssymb}
\usepackage{graphicx}
\usepackage[a4paper, margin=1in]{geometry}

\usepackage{xcolor,sectsty}
\definecolor{astral}{RGB}{46,116,181}
\subsectionfont{\color{astral}}
\sectionfont{\color{astral}}


\linespread{1.15}
\title{\Huge\color{astral}\textbf{Plano de Aula}}
\author{Paulo Brandão}
\date{Maio de 2017}

\begin{document}

\maketitle

\section{Dados de Identificação}

\noindent \textbf{Instituição}: Instituto de Física - Universidade Federal de Alagoas.

\noindent \textbf{Disciplina}: Física da Matéria Condensada / Eletromagnetismo.

\noindent \textbf{Tema}: Propriedades Eletromagnéticas da Matéria.

\noindent \textbf{Professor}: Paulo Cesar Aguiar Brandão Filho.

\noindent \textbf{Tempo total de aula}: 50 minutos.

\noindent \textbf{Data}: 00/00/2018.

\section{Objetivos}

\begin{itemize}
    \item Compreender a base teórica do formalismo das equações de Maxwell aplicada aos materiais.
    \item Analisar a base adquirida na resolução de problemas práticos.
\end{itemize}

\section{Conteúdos}

\begin{enumerate}
    \item Introdução
    \item Equações de Maxwell num Meio Dielétrico
    \item Resposta Dielétrica Linear da Matéria
    \item Dependência da Frequência no Tensor Dielétrico
    \item Propagação de uma Onda Eletromagnética num Meio Dispersivo
    \item Natureza Física da Banda de Absorção Eletrônica
    \item Conclusões
\end{enumerate}

\section{Metodologia e Recursos}

A aula será de caráter expositivo e dialógico, tendo como recursos materiais: quadro, marcador, resumo da aula (anexo 1) e lista de exercícios (anexo 2).

Começamos a discussão na seção 2 introduzindo o formalismo necessário para a descrição da interação radiação-matéria de um ponto de vista clássico macroscópico através do uso das equações de Maxwell. Uma aproximação linear será discutida e introduzimos a importante quantidade chamada de função dielétrica, que irá ser estudada no decorrer de toda a aula. A seção 3 se concentra na resposta linear e nos problemas de causalidade impostos pelas relações de Kramers-Kroning. Uma discussão importante sobre dispersão temporal e espacial também é considerada. Na seção 4 estudaremos o caso particular da dependência da função dielétrica com a frequência do campo externo e derivaremos uma relação muito importante para a parte real da função dielétrica. A seção 5 descreve brevemente o comportamento de uma onda eletromagnética quando a mesma se propaga num meio dispersivo. Na seção 6 discutiremos a origem física das ressonâncias existentes num espectro de absorção, caracterizado pela parte imaginária da função dielétrica. Apenas trataremos a contribuição eletrônica da função dielétrica. Concluímos a aula na seção 7 com uma discussão breve sobre as outras formas de ressonâncias existentes nos materiais. O conteúdo programático foi inspirado nas referências [1,2,3,4].

A aula ocorrerá através da promoção contínua da participação do aluno com o que está sendo discutido, estimulando o resgate daquele conhecimento prévio que o mesmo já possa ter adquirido.

\section{Avaliação}

A avaliação será realizada através da participação em aula durante a exposição do tema e do diagnóstico da resolução dos exercícios da lista do anexo II. 

\section{Referências}

\noindent [1] D. L. Mills \textit{Nonlinear Optics - Basic Concepts} (Springer, 1998).

\noindent [2] R. P. Feynman, R. B. Leighton, M. Sands, \textit{The Feynman Lectures on Physics}, Vol. 2 (Addison Wesley, 1977).

\noindent [3] J. S. Blakemore, \textit{Solid State Physics}, 2 ed. (Cambridge University Press, 1985).

\noindent [4] N. W. Ashcroft, N. David Mermin, \textit{Solid State Physics} (Saunders College Publishing, 1976).


\end{document}
