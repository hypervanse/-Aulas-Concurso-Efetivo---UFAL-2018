\documentclass{article}
\usepackage[utf8]{inputenc}
\usepackage{amsmath}
\usepackage{indentfirst}
\usepackage{amssymb}
\usepackage{graphicx}
\usepackage[a4paper, margin=1in]{geometry}

\usepackage{xcolor,sectsty}
\definecolor{astral}{RGB}{46,116,181}
\subsectionfont{\color{astral}}
\sectionfont{\color{astral}}

\linespread{1.15}
\title{\color{astral}\textbf{Lista de Exercícios} \\ \textbf{Propriedades Eletromagnéticas da Matéria}}
\author{Paulo Brandão}
\date{Maio de 2017}

\begin{document}

\maketitle

\noindent 1. Uma esfera com raio $R$ é fabricada de um material ferroelétrico, e tem uma polarização espacial uniforme $\mathbf{P} = \hat{\mathbf{z}}P_0$ paralela à direção $z$. Encontre $\mathbf{D}$ e $\mathbf{E}$ dentro e fora do material. 

\vspace{1cm}

\noindent 2. Uma carga pontual $Q$ é colocada na origem de um dielétrico isotrópico não-linear. Assim, tanto $\mathbf{E}$ como $\mathbf{D}$ estão na direção radial, por simetria. Temos, com $D_r$ e $E_r$ os campos radiais $D_r = \varepsilon_0( E_r + \chi^{(1)}E_r + \chi^{(3)}E_r^3 )$. Assuma que $\chi^{(3)}>0$ e discuta o comportamento de $E_r$ e $D_r$, com atenção para os limites $r\rightarrow \infty$ e $r\rightarrow 0$.

\vspace{1cm}

\noindent 3. Demonstre a Eq. (38) das notas de aula.

\end{document}
