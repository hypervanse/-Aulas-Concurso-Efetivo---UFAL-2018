\documentclass{article}
\usepackage[utf8]{inputenc}
\usepackage{amsmath}
\usepackage{indentfirst}
\usepackage{amssymb}
\usepackage{graphicx}
\usepackage[a4paper, margin=1in]{geometry}

\usepackage{xcolor,sectsty}
\definecolor{astral}{RGB}{46,116,181}
\subsectionfont{\color{astral}}
\sectionfont{\color{astral}}


\linespread{1.15}
\title{\Huge\color{astral}\textbf{Plano de Aula}}
\author{Paulo Brandão}
\date{Maio de 2017}

\begin{document}

\maketitle

\section{Dados de Identificação}

\noindent \textbf{Instituição}: Instituto de Física - Universidade Federal de Alagoas.

\noindent \textbf{Disciplina}: Mecânica Quântica.

\noindent \textbf{Tema}: Teoria Quântica do Espalhamento

\noindent \textbf{Professor}: Paulo Cesar Aguiar Brandão Filho.

\noindent \textbf{Tempo total de aula}: 50 minutos.

\noindent \textbf{Data}: 00/00/2018.

\section{Objetivos}

\begin{itemize}
    \item Compreender a base matemática e física da teoria quântica do espalhamento na aproximação de Born.
    \item Analisar a base adquirida na resolução de problemas práticos de espalhamento.
\end{itemize}

\section{Conteúdos}

\begin{enumerate}
    \item Introdução
        \subitem 1.1 Descrição de Experimentos de Espalhamento
        \subitem 1.2 Diferentes Tipos de Espalhamento
        \subitem 1.3 Quantidades Observáveis
    \item Formalismo do Espalhamento Quântico
    \item Aproximação de Born
        \subitem 3.1 Relação entre $f$ e a Seção de Choque
    \item Aplicações
        \subitem 4.1 Potencial ``Esfera Rígida''
        \subitem 4.2 Potencial de Yukawa
        \subitem 4.3 Potencial de Coulomb (Espalhamento de Rutherford
    \item Aproximação de Born Para Altas Ordens
    \item Conclusão
    \item Apêndice: Solução da Equação $(\nabla^2 + k^2)\psi = Q$
\end{enumerate}

\section{Metodologia e Recursos}

A aula será de caráter expositivo e dialógico, tendo como recursos materiais: quadro, marcador, resumo da aula (anexo 1) e lista de exercícios (anexo 2).

A importância do estudo do espalhamento em sistemas quânticos é enfatizada inicialmente na seção 1 e em suas subseções. Descreveremos a preparação de um sistema para o estudo do espalhamento, os diferentes tipos de espalhamento que podem ocorrer num determinado sistema e que tipo de grandeza é geralmente medida no experimento. 

Na seção 2 começamos o formalismo matemático da teoria utilizando como base a equação de Schrödinger independente do tempo. O objetivo dessa seção é colocar a equação numa forma que será útil nas etapas posteriores.

A seção 3 apresenta os resultados principais da teoria de perturbação em primeira ordem, ou aproximação de Born, para a solução do problema do espalhamento. Relacionamos a amplitude de espalhamento, que depende do potencial espalhador, com a seção de choque, que representa a grandeza medida num laboratório.

Três exemplos de espalhamento por potenciais específicos são apresentados na seção 4 e mais outro na lista de exercícios. O objetivo dessa seção é mostrar ao aluno a aplicação direta das fórmulas desenvolvidas na seção anterior e como interpretar um gráfico da seção de choque diferencial.

Ordens mais altas na expansão de Born são apresentadas na seção 5 e suas consequências serão discutidas. As conclusões são descritas na seção 6 e o apêndice 7 foi incluso para o estudante verificar com detalhe a forma integral da equação de Schrödinger que foi utilizada na formulação da teoria. O conteúdo programático foi inspirado nas referências [1,2,3,4].

A aula ocorrerá através da promoção contínua da participação do aluno com o que está sendo discutido, estimulando o resgate daquele conhecimento prévio que o mesmo já possa ter adquirido.

\section{Avaliação}

A avaliação será realizada através da participação em aula durante a exposição do tema e do diagnóstico da resolução dos exercícios da lista do anexo II. 

\section{Referências}

\noindent [1] David J. Griffiths, \textit{Introduction to Quantum Mechanics}, 2a Ed. (Pearson Prentice Hall, 2005).

\noindent [2] J. J. Sakurai, \textit{Modern Quantum Mechanics}, Revised Edition. (Addison-Wesley Publishing Company, Inc, 1994).

\noindent [3] R. Shankar, \textit{Principles of Quantum Mechanics} (Plenum Press, 1994).

\noindent [2] John R. Taylor \textit{Scattering Theory: The Quantum Theory on Nonrelativistic Collisions} (Wiley, 1972).



\end{document}
