\documentclass{article}
\usepackage[utf8]{inputenc}
\usepackage{amsmath}
\usepackage{indentfirst}
\usepackage{amssymb}
\usepackage{graphicx}
\usepackage[a4paper, margin=1in]{geometry}

\usepackage{xcolor,sectsty}
\definecolor{astral}{RGB}{46,116,181}
\subsectionfont{\color{astral}}
\sectionfont{\color{astral}}


\linespread{1.15}
\title{\Huge\color{astral}\textbf{Plano de Aula}}
\author{Paulo Brandão}
\date{Maio de 2017}

\begin{document}

\maketitle

\section{Dados de Identificação}

\noindent \textbf{Instituição}: Instituto de Física - Universidade Federal de Alagoas.

\noindent \textbf{Disciplina}: Mecânica Quântica.

\noindent \textbf{Tema}: Métodos Perturbativos em Mecânica Quântica.

\noindent \textbf{Professor}: Paulo Cesar Aguiar Brandão Filho.

\noindent \textbf{Tempo total de aula}: 50 minutos.

\noindent \textbf{Data}: 00/00/2018.

\section{Objetivos}

\begin{itemize}
    \item Compreender a base matemática e física da teoria de perturbação.
    \item Analisar a base adquirida na resolução de problemas de teoria quântica.
\end{itemize}

\section{Conteúdos}

\begin{enumerate}
    \item Introdução
    \item As Três Etapas de um Método Perturbativo
    \item Um Exemplo Simples
    \item A Equação de Schrödinger Independente do Tempo
    \item Método Perturbativo Não-Degenerado
    \item Método Perturbativo Degenerado
    \item Conclusões
\end{enumerate}

\section{Metodologia e Recursos}

A aula será de caráter expositivo e dialógico, tendo como recursos materiais: quadro, marcador, resumo da aula (anexo 1) e lista de exercícios (anexo 2).

A necessidade de se estudar métodos de perturbação é comentada na seção introdutória 1. Na seção 2, discutimos os três passos essenciais da teoria de perturbação. Aplicamos esses passos na resolução de um problema algébrico na seção 3 com o objetivo de nivelar o estudante com relação às dificuldades e sutilezas que podem ocorrer durante a solução de um problema perturbativo. Na seção 4 formularemos o problema perturbativo com relação à equação de Schrödinger independente do tempo. Seguiremos os passos descritos na seção 2 para finalmente resolver a equação de Schrödinger na seção 5 assumindo que os estados não-perturbados não possuam degenerescência (teoria de perturbação não-degenerada). A aula termina na seção 6 com o desenvolvimento da teoria de perturbação para o caso particular onde os estados não-perturbados são degenerados. Apresentamos as conclusões na seção 7.  O conteúdo programático foi inspirado nas referências [1,2,3,4].

A aula ocorrerá através da promoção contínua da participação do aluno com o que está sendo discutido, estimulando o resgate daquele conhecimento prévio que o mesmo já possa ter adquirido.

\section{Avaliação}

A avaliação será realizada através da participação em aula durante a exposição do tema e do diagnóstico da resolução dos exercícios da lista do anexo II. 

\section{Referências}

\noindent [1] Carl. M. Bender and S. A. Orszag \textit{Advanced Mathematical Methods for Scientists and Engineers} (Springer, 1999).

\noindent [2] David J. Griffiths, \textit{Introduction to Quantum Mechanics}, 2a Ed. (Pearson Prentice Hall, 2005).

\noindent [3] J. J. Sakurai, \textit{Modern Quantum Mechanics}, Revised Edition. (Addison-Wesley Publishing Company, Inc, 1994).

\noindent [4] R. Shankar, \textit{Principles of Quantum Mechanics} (Plenum Press, 1994).


\end{document}
