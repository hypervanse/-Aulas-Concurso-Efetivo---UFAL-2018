\documentclass{article}
\usepackage[utf8]{inputenc}
\usepackage{amsmath}
\usepackage{indentfirst}
\usepackage{amssymb}
\usepackage{graphicx}
\usepackage[a4paper, margin=1in]{geometry}

\usepackage{xcolor,sectsty}
\definecolor{astral}{RGB}{46,116,181}
\subsectionfont{\color{astral}}
\sectionfont{\color{astral}}

\linespread{1.15}
\title{\color{astral}\textbf{Lista de Exercícios} \\ \textbf{Métodos Perturbativos em Mecânica Quântica}}
\author{Paulo Brandão}
\date{Maio de 2017}

\begin{document}

\maketitle


\noindent 1. Utilize a teoria da perturbação de segunda ordem para encontrar aproximações para as raízes das seguintes equações:

\vspace{1cm}

\noindent (a) $x^2 + x + 6\varepsilon = 0$;\\
\noindent (b) $x^3 - \varepsilon x - 1 = 0$;\\
\noindent (c) $x^3 + \varepsilon x^2 - x = 0$.

\vspace{1cm}

\noindent 2. Encontre uma fórmula genérica para os coeficientes $a_n$ da série (4) das notas de aula e, com isso, calcule o raio de convergência da série.

\vspace{1cm}

\noindent 3. Encontre uma fórmula genérica para os coeficientes $a_n$ da série (6) das notas de aula e, com isso, calcule o raio de convergência da série.

\vspace{1cm}

\noindent 4. Suponha que colocamos uma perturbação do tipo delta dirac no centro do poço quadrado infinito:
\begin{equation}
    W = \varepsilon\delta(x-a/2),
\end{equation}
onde $\varepsilon$ é uma constante. (a) Encontre as correções de primeira ordem da energia. Explique porque as energias não são perturbadas para $n$ par. (b) Encontre os primeiros termos não-nulos na expansão da função de de onda corrigida de primeira ordem.

\vspace{1cm}

\noindent 5. Considere uma partícula de massa $m$ que esteja livre para se mover numa região unidimensional de comprimento $L$ que se fecha nela mesma. (a) Mostre que os estados estacionários podem ser escritos na forma
\begin{equation}
    \psi_n (x) = \frac{1}{\sqrt{L}}e^{2\pi i n x/L}
\end{equation} 
com $-L/2<x<L/2$,  $n = 0,\pm 1,\pm2 ,...$ e energias
\begin{equation}
    E_n = \frac{2}{m}\left( \frac{n\pi\hbar}{L} \right)^2.
\end{equation}
Note que todas as energias, com exceção do estado fundamentel $n = 0$, são duplamente degeneradas. (b) Encontre a correção em primeira ordem da energia se introduzirmos a perturbação
\begin{equation}
    W = -V_{0}e^{-x^2 / a^2}.
\end{equation}




\end{document}
